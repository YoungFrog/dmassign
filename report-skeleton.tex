%%% You must provide \facteurThExe


%%% Headers
\documentclass[a4paper,landscape,french]{article}
\usepackage[margin=1.5cm]{geometry}
\usepackage{babel}
\usepackage[T1]{fontenc}
\usepackage[utf8]{inputenc}
\usepackage{graphicx}
\usepackage{array}
\usepackage{longtable}
\usepackage{booktabs}
\usepackage{tabularx}
\usepackage{xstring}
\usepackage{tabu}
\usepackage{url}
\usepackage{genmpage}
\usepackage{varwidth} % defines the V column type.
\usepackage{epstopdf}
\usepackage[usenames,dvipsnames]{color}
\usepackage[colorlinks,breaklinks,bookmarks]{hyperref}

%%% Helper macros.
\let\narrowragged\raggedright % avoid the effect of \narrowragged :
                              % since we use varwidth in a table (V
                              % column type), it can
                              % lead to uneven line lengths accros
                              % different table lines (which are typeset
                              % in different varwidth envir.)
\newenvironment{teacher}[2][]{
\phantomsection
\addcontentsline{toc}{subsection}{#2}
\def\score{\IfStrEq{}{#1}{}{~--~Score: #1}}%made a macro for clarity.
\tabular{|V{10cm}*{6}{l}|}
%\firsthline
\multicolumn{6}{l}{\bf #2 -- Enseignements \score}\\
\hline
\textsl{Cours} &\textsl{Groupe} &\textsl{Th} &\textsl{Exe} &\textsl{TP}& \textsl{Total} & \textsl{Programmé}\\
\hline
}{
\lasthline
\endtabular
}
\newenvironment{teacherhoraire}[1]{
\tabular{|V{10cm}|l|}
\firsthline
\multicolumn{2}{|l|}{\bf #1 -- Horaire}\\
\hline
\hline
\textsl{Horaire} &\textsl{T\^ache} \\
\hline
}{
\lasthline
\endtabular
}
\newenvironment{teacheradmin}[1]{
\tabular{|V{10cm}|l|}
\firsthline
\multicolumn{2}{|l|}{\bf #1 -- T\^aches administratives}\\
\hline
\hline
\textsl{T\^ache} &\textsl{Rating}\\
\hline
}{
\lasthline
\endtabular
}
\makeatletter
\newcommand\nicegraph[6]{\setlength{\unitlength}{1cm}
\raisebox{-\height}{\hbox{\begin{picture}(2.75,3.75)
\multiput(0,0.4)(0,.2){16}{\line(1,0){2.25}}
\put(2.3,0.3){\small $0$}
\put(2.3,0.55){\tiny $10$}
\put(2.3,0.75){\tiny $20$}
\put(2.3,0.95){\tiny $30$}
\put(2.3,1.15){\tiny $40$}
\put(2.3,1.3){\small $50$}
\put(2.3,1.55){\tiny $60$}
\put(2.3,1.75){\tiny $70$}
\put(2.3,1.95){\tiny $80$}
\put(2.3,2.15){\tiny $90$}
\put(2.3,2.3){\small $100$}
\put(2.3,2.55){\tiny $110$}
\put(2.3,2.75){\tiny $120$}
\put(2.3,2.95){\tiny $130$}
\put(2.3,3.15){\tiny $140$}
\put(2.3,3.3){\small $150$}
\put(0.2,0){\scriptsize SE}
\put(0.95,0){\scriptsize BA}
\put(1.7,0){\scriptsize MA}
\put(0.25, 0.4){\color{BlueViolet}\rule{.25cm}{\dimexpr #1 cm/50\relax}} % SEth
\put(0.25, \expandafter\strip@pt\dimexpr #1pt/50 + 2pt/5\relax){\color{RawSienna}\rule{.25cm}{\dimexpr #2 cm/50\relax}} % SEtp+ex (en unités eq_th)
\put(1.0,0.4){\color{BlueViolet}\rule{.25cm}{\dimexpr #3 cm/50\relax}} % BAth
\put(1.0, \expandafter\strip@pt\dimexpr #3pt/50 + 2pt/5\relax){\color{RawSienna}\rule{.25cm}{\dimexpr #4 cm/50\relax}} % BAtp+ex (en unités eq_th)
\put(1.75,0.4){\color{BlueViolet}\rule{.25cm}{\dimexpr #5 cm/50\relax}} % MAth
\put(1.75,\expandafter\strip@pt\dimexpr #5pt/50 + 2pt/5\relax){\color{RawSienna}\rule{.25cm}{\dimexpr #6 cm/50\relax}}
\end{picture}}}}
\makeatother

%%% Actual document
\title{D\'epartement de Math\'ematique~: r\'epartition des enseignements~\anneeaca}
\author{Coll\`ege d'Enseignement / Commission P\'edagogique}
\raggedright % this avoids TeX trying to avoid underful boxes
\setcounter{tocdepth}{1}

\begin{document}
\maketitle
\section{Remarques introductives}
\begin{minipage}{0.6\linewidth}[keepparindent]
  Ce document de travail interne et pr\'eliminaire concerne la répartition des tâches pédagogiques au sein du département de Mathématiques.
\par\medskip

Deux unit\'es sont utilis\'ees, ``h-th'' (pour ``heure de th\'eorie'') et 
``h-exe'' (pour ``heure d'exercice ou de travail personnel''). Dans la r\'epartition, les totaux sont calculés selon la règle :
1 h-th \'equivaut \`a \facteurThExe h-exe.
\par\medskip

%% Pourriez-vous consulter ce document en faisant particuli\`erement attention aux points suivants~:
%% \begin{itemize}
%% \item Les personnes absentes, ou arrivant, en \anneeaca.
%% \item Les enseignements qui ne figurent pas, ou qui ne devraient pas figurer.
%% \item Les enseignements de MA pour lesquels devraient \^etre comptabilis\'es des exercices, sans que
%%        cela apparaisse.
%% \end{itemize}

\begin{center}
\begin{tabular}{ll}
\toprule
\multicolumn{2}{c}{Légende pour les graphes à gauche}\\
\midrule
SE & enseignement de service\\
BA & enseignement du BA math\\
MA & enseignement des MA math, stat ou actu\\\midrule
Rouge & Exe (Séances d'exercices) et TP (\og Travaux personnels\fg{})\\
Bleu & Théorie\\
\bottomrule
\end{tabular}
\end{center}
\end{minipage}

\tableofcontents

\section{R\'epartition pr\'eliminaire}
\shorthandoff{;!:?}
%% @@@DATA@@@ %% This line will be ignored.

% \section{Beau graphique}
% \begin{center}
% \includegraphics[width=10cm,angle=-90]{load}
% \end{center}

\end{document}

%%% Local Variables:
%%% mode: latex
%%% TeX-master: t
%%% End:
